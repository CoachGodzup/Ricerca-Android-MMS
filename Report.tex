%% LyX 2.0.2 created this file.  For more info, see http://www.lyx.org/.
%% Do not edit unless you really know what you are doing.
\documentclass[10pt,a4paper]{article}
\usepackage[T1]{fontenc}
\usepackage[latin1]{inputenc}
\pagestyle{empty}
\usepackage{array}
\usepackage{amsmath}
\usepackage{amssymb}
\usepackage{graphicx}

\makeatletter

%%%%%%%%%%%%%%%%%%%%%%%%%%%%%% LyX specific LaTeX commands.
\pdfpageheight\paperheight
\pdfpagewidth\paperwidth

%% Because html converters don't know tabularnewline
\providecommand{\tabularnewline}{\\}
%% A simple dot to overcome graphicx limitations
\newcommand{\lyxdot}{.}


%%%%%%%%%%%%%%%%%%%%%%%%%%%%%% User specified LaTeX commands.
% VDE Template for EUSAR Papers
% Provided by Barbara Lang und Siegmar Lampe
% University of Bremen, January 2002
% English version by Jens Fischer
% German Aerospace Center (DLR), December 2005
% Additional modifications by Matthias Wei{\ss}
% FGAN, January 2009

%-----------------------------------------------------------------------------
% Type of publication

%-----------------------------------------------------------------------------
% Other packets: Most packets may be downloaded from www.dante.de and
% "tcilatex.tex" can be found at (December 2005):
% http://www.mackichan.com/techtalk/v30/UsingFloat.htm
% Not all packets are necessarily needed:
%\usepackage{ngerman} % in german language if required
\usepackage[nooneline,bf]{caption}% Figure descriptions from left margin
\usepackage{times}\usepackage{multicol}\usepackage{epsfig}% Macros for Scientific Word 3.0 documents saved with the LaTeX filter.
%Copyright (C) 1994-97 TCI Software Research, Inc.
\typeout{TCILATEX Macros for Scientific Word 3.0 <05 August 1998>.}
\typeout{NOTICE:  This macro file is NOT proprietary and may be 
freely copied and distributed.}
%
\makeatletter
%
%%%%%%%%%%%%%%%%%%%%%%
% macros for time
\newcount\@hour\newcount\@minute\chardef\@x10\chardef\@xv60
\def\tcitime{
\def\@time{%
  \@minute\time\@hour\@minute\divide\@hour\@xv
  \ifnum\@hour<\@x 0\fi\the\@hour:%
  \multiply\@hour\@xv\advance\@minute-\@hour
  \ifnum\@minute<\@x 0\fi\the\@minute
  }}%

%%%%%%%%%%%%%%%%%%%%%%
% macro for hyperref
\@ifundefined{hyperref}{\def\hyperref#1#2#3#4{#2\ref{#4}#3}}{}

% macro for external program call
\@ifundefined{qExtProgCall}{\def\qExtProgCall#1#2#3#4#5#6{\relax}}{}
%%%%%%%%%%%%%%%%%%%%%%
%
% macros for graphics
%
\def\FILENAME#1{#1}%
%
\def\QCTOpt[#1]#2{%
  \def\QCTOptB{#1}
  \def\QCTOptA{#2}
}
\def\QCTNOpt#1{%
  \def\QCTOptA{#1}
  \let\QCTOptB\empty
}
\def\Qct{%
  \@ifnextchar[{%
    \QCTOpt}{\QCTNOpt}
}
\def\QCBOpt[#1]#2{%
  \def\QCBOptB{#1}
  \def\QCBOptA{#2}
}
\def\QCBNOpt#1{%
  \def\QCBOptA{#1}
  \let\QCBOptB\empty
}
\def\Qcb{%
  \@ifnextchar[{%
    \QCBOpt}{\QCBNOpt}
}
\def\PrepCapArgs{%
  \ifx\QCBOptA\empty
    \ifx\QCTOptA\empty
      {}%
    \else
      \ifx\QCTOptB\empty
        {\QCTOptA}%
      \else
        [\QCTOptB]{\QCTOptA}%
      \fi
    \fi
  \else
    \ifx\QCBOptA\empty
      {}%
    \else
      \ifx\QCBOptB\empty
        {\QCBOptA}%
      \else
        [\QCBOptB]{\QCBOptA}%
      \fi
    \fi
  \fi
}
\newcount\GRAPHICSTYPE
%\GRAPHICSTYPE 0 is for TurboTeX
%\GRAPHICSTYPE 1 is for DVIWindo (PostScript)
%%%(removed)%\GRAPHICSTYPE 2 is for psfig (PostScript)
\GRAPHICSTYPE=\z@
\def\GRAPHICSPS#1{%
 \ifcase\GRAPHICSTYPE%\GRAPHICSTYPE=0
   \special{ps: #1}%
 \or%\GRAPHICSTYPE=1
   \special{language "PS", include "#1"}%
%%%\or%\GRAPHICSTYPE=2
%%%  #1%
 \fi
}%
%
\def\GRAPHICSHP#1{\special{include #1}}%
%
% \graffile{ body }                                  %#1
%          { contentswidth (scalar)  }               %#2
%          { contentsheight (scalar) }               %#3
%          { vertical shift when in-line (scalar) }  %#4
\def\graffile#1#2#3#4{%
%%% \ifnum\GRAPHICSTYPE=\tw@
%%%  %Following if using psfig
%%%  \@ifundefined{psfig}{\input psfig.tex}{}%
%%%  \psfig{file=#1, height=#3, width=#2}%
%%% \else
  %Following for all others
  % JCS - added BOXTHEFRAME, see below
    \bgroup
    \leavevmode
    \@ifundefined{bbl@deactivate}{\def~{\string~}}{\activesoff}
    \raise -#4 \BOXTHEFRAME{%
        \hbox to #2{\raise #3\hbox to #2{\null #1\hfil}}}%
    \egroup
}%
%
% A box for drafts
\def\draftbox#1#2#3#4{%
 \leavevmode\raise -#4 \hbox{%
  \frame{\rlap{\protect\tiny #1}\hbox to #2%
   {\vrule height#3 width\z@ depth\z@\hfil}%
  }%
 }%
}%
%
\newcount\draft
\draft=\z@
\let\nographics=\draft
\newif\ifwasdraft
\wasdraftfalse

%  \GRAPHIC{ body }                                  %#1
%          { draft name }                            %#2
%          { contentswidth (scalar)  }               %#3
%          { contentsheight (scalar) }               %#4
%          { vertical shift when in-line (scalar) }  %#5
\def\GRAPHIC#1#2#3#4#5{%
 \ifnum\draft=\@ne\draftbox{#2}{#3}{#4}{#5}%
  \else\graffile{#1}{#3}{#4}{#5}%
  \fi
 }%
%
\def\addtoLaTeXparams#1{%
    \edef\LaTeXparams{\LaTeXparams #1}}%
%
% JCS -  added a switch BoxFrame that can 
% be set by including X in the frame params.
% If set a box is drawn around the frame.

\newif\ifBoxFrame \BoxFramefalse
\newif\ifOverFrame \OverFramefalse
\newif\ifUnderFrame \UnderFramefalse

\def\BOXTHEFRAME#1{%
   \hbox{%
      \ifBoxFrame
         \frame{#1}%
      \else
         {#1}%
      \fi
   }%
}


\def\doFRAMEparams#1{\BoxFramefalse\OverFramefalse\UnderFramefalse\readFRAMEparams#1\end}%
\def\readFRAMEparams#1{%
 \ifx#1\end%
  \let\next=\relax
  \else
  \ifx#1i\dispkind=\z@\fi
  \ifx#1d\dispkind=\@ne\fi
  \ifx#1f\dispkind=\tw@\fi
  \ifx#1t\addtoLaTeXparams{t}\fi
  \ifx#1b\addtoLaTeXparams{b}\fi
  \ifx#1p\addtoLaTeXparams{p}\fi
  \ifx#1h\addtoLaTeXparams{h}\fi
  \ifx#1X\BoxFrametrue\fi
  \ifx#1O\OverFrametrue\fi
  \ifx#1U\UnderFrametrue\fi
  \ifx#1w
    \ifnum\draft=1\wasdrafttrue\else\wasdraftfalse\fi
    \draft=\@ne
  \fi
  \let\next=\readFRAMEparams
  \fi
 \next
 }%
%
%Macro for In-line graphics object
%   \IFRAME{ contentswidth (scalar)  }               %#1
%          { contentsheight (scalar) }               %#2
%          { vertical shift when in-line (scalar) }  %#3
%          { draft name }                            %#4
%          { body }                                  %#5
%          { caption}                                %#6


\def\IFRAME#1#2#3#4#5#6{%
      \bgroup
      \let\QCTOptA\empty
      \let\QCTOptB\empty
      \let\QCBOptA\empty
      \let\QCBOptB\empty
      #6%
      \parindent=0pt%
      \leftskip=0pt
      \rightskip=0pt
      \setbox0 = \hbox{\QCBOptA}%
      \@tempdima = #1\relax
      \ifOverFrame
          % Do this later
          \typeout{This is not implemented yet}%
          \show\HELP
      \else
         \ifdim\wd0>\@tempdima
            \advance\@tempdima by \@tempdima
            \ifdim\wd0 >\@tempdima
               \textwidth=\@tempdima
               \setbox1 =\vbox{%
                  \noindent\hbox to \@tempdima{\hfill\GRAPHIC{#5}{#4}{#1}{#2}{#3}\hfill}\\%
                  \noindent\hbox to \@tempdima{\parbox[b]{\@tempdima}{\QCBOptA}}%
               }%
               \wd1=\@tempdima
            \else
               \textwidth=\wd0
               \setbox1 =\vbox{%
                 \noindent\hbox to \wd0{\hfill\GRAPHIC{#5}{#4}{#1}{#2}{#3}\hfill}\\%
                 \noindent\hbox{\QCBOptA}%
               }%
               \wd1=\wd0
            \fi
         \else
            %\show\BBB
            \ifdim\wd0>0pt
              \hsize=\@tempdima
              \setbox1 =\vbox{%
                \unskip\GRAPHIC{#5}{#4}{#1}{#2}{0pt}%
                \break
                \unskip\hbox to \@tempdima{\hfill \QCBOptA\hfill}%
              }%
              \wd1=\@tempdima
           \else
              \hsize=\@tempdima
              \setbox1 =\vbox{%
                \unskip\GRAPHIC{#5}{#4}{#1}{#2}{0pt}%
              }%
              \wd1=\@tempdima
           \fi
         \fi
         \@tempdimb=\ht1
         \advance\@tempdimb by \dp1
         \advance\@tempdimb by -#2%
         \advance\@tempdimb by #3%
         \leavevmode
         \raise -\@tempdimb \hbox{\box1}%
      \fi
      \egroup%
}%
%
%Macro for Display graphics object
%   \DFRAME{ contentswidth (scalar)  }               %#1
%          { contentsheight (scalar) }               %#2
%          { draft label }                           %#3
%          { name }                                  %#4
%          { caption}                                %#5
\def\DFRAME#1#2#3#4#5{%
 \begin{center}
     \let\QCTOptA\empty
     \let\QCTOptB\empty
     \let\QCBOptA\empty
     \let\QCBOptB\empty
     \ifOverFrame 
        #5\QCTOptA\par
     \fi
     \GRAPHIC{#4}{#3}{#1}{#2}{\z@}
     \ifUnderFrame 
        \nobreak\par\nobreak#5\QCBOptA
     \fi
 \end{center}%
 }%
%
%Macro for Floating graphic object
%   \FFRAME{ framedata f|i tbph x F|T }              %#1
%          { contentswidth (scalar)  }               %#2
%          { contentsheight (scalar) }               %#3
%          { caption }                               %#4
%          { label }                                 %#5
%          { draft name }                            %#6
%          { body }                                  %#7
\def\FFRAME#1#2#3#4#5#6#7{%
 %If float.sty loaded and float option is 'h', change to 'H'  (gp) 1998/09/05
  \@ifundefined{floatstyle}
    {%floatstyle undefined (and float.sty not present), no change
     \begin{figure}[#1]%
    }
    {%floatstyle DEFINED
	 \ifx#1h%Only the h parameter, change to H
      \begin{figure}[H]%
	 \else
      \begin{figure}[#1]%
	 \fi
	}
  \let\QCTOptA\empty
  \let\QCTOptB\empty
  \let\QCBOptA\empty
  \let\QCBOptB\empty
  \ifOverFrame
    #4
    \ifx\QCTOptA\empty
    \else
      \ifx\QCTOptB\empty
        \caption{\QCTOptA}%
      \else
        \caption[\QCTOptB]{\QCTOptA}%
      \fi
    \fi
    \ifUnderFrame\else
      \label{#5}%
    \fi
  \else
    \UnderFrametrue%
  \fi
  \begin{center}\GRAPHIC{#7}{#6}{#2}{#3}{\z@}\end{center}%
  \ifUnderFrame
    #4
    \ifx\QCBOptA\empty
      \caption{}%
    \else
      \ifx\QCBOptB\empty
        \caption{\QCBOptA}%
      \else
        \caption[\QCBOptB]{\QCBOptA}%
      \fi
    \fi
    \label{#5}%
  \fi
  \end{figure}%
 }%
%
%
%    \FRAME{ framedata f|i tbph x F|T }              %#1
%          { contentswidth (scalar)  }               %#2
%          { contentsheight (scalar) }               %#3
%          { vertical shift when in-line (scalar) }  %#4
%          { caption }                               %#5
%          { label }                                 %#6
%          { name }                                  %#7
%          { body }                                  %#8
%
%    framedata is a string which can contain the following
%    characters: idftbphxFT
%    Their meaning is as follows:
%             i, d or f : in-line, display, or floating
%             t,b,p,h   : LaTeX floating placement options
%             x         : fit contents box to contents
%             F or T    : Figure or Table. 
%                         Later this can expand
%                         to a more general float class.
%
%
\newcount\dispkind%

\def\makeactives{
  \catcode`\"=\active
  \catcode`\;=\active
  \catcode`\:=\active
  \catcode`\'=\active
  \catcode`\~=\active
}
\bgroup
   \makeactives
   \gdef\activesoff{%
      \def"{\string"}
      \def;{\string;}
      \def:{\string:}
      \def'{\string'}
      \def~{\string~}
      %\bbl@deactivate{"}%
      %\bbl@deactivate{;}%
      %\bbl@deactivate{:}%
      %\bbl@deactivate{'}%
    }
\egroup

\def\FRAME#1#2#3#4#5#6#7#8{%
 \bgroup
 \ifnum\draft=\@ne
   \wasdrafttrue
 \else
   \wasdraftfalse%
 \fi
 \def\LaTeXparams{}%
 \dispkind=\z@
 \def\LaTeXparams{}%
 \doFRAMEparams{#1}%
 \ifnum\dispkind=\z@\IFRAME{#2}{#3}{#4}{#7}{#8}{#5}\else
  \ifnum\dispkind=\@ne\DFRAME{#2}{#3}{#7}{#8}{#5}\else
   \ifnum\dispkind=\tw@
    \edef\@tempa{\noexpand\FFRAME{\LaTeXparams}}%
    \@tempa{#2}{#3}{#5}{#6}{#7}{#8}%
    \fi
   \fi
  \fi
  \ifwasdraft\draft=1\else\draft=0\fi{}%
  \egroup
 }%
%
% This macro added to let SW gobble a parameter that
% should not be passed on and expanded. 

\def\TEXUX#1{"texux"}

%
% Macros for text attributes:
%
\def\BF#1{{\bf {#1}}}%
\def\NEG#1{\leavevmode\hbox{\rlap{\thinspace/}{$#1$}}}%
%
%%%%%%%%%%%%%%%%%%%%%%%%%%%%%%%%%%%%%%%%%%%%%%%%%%%%%%%%%%%%%%%%%%%%%%%%
%
%
% macros for user - defined functions
\def\limfunc#1{\mathop{\rm #1}}%
\def\func#1{\mathop{\rm #1}\nolimits}%
% macro for unit names
\def\unit#1{\mathop{\rm #1}\nolimits}%

%
% miscellaneous 
\long\def\QQQ#1#2{%
     \long\expandafter\def\csname#1\endcsname{#2}}%
\@ifundefined{QTP}{\def\QTP#1{}}{}
\@ifundefined{QEXCLUDE}{\def\QEXCLUDE#1{}}{}
\@ifundefined{Qlb}{\def\Qlb#1{#1}}{}
\@ifundefined{Qlt}{\def\Qlt#1{#1}}{}
\def\QWE{}%
\long\def\QQA#1#2{}%
\def\QTR#1#2{{\csname#1\endcsname #2}}%(gp) Is this the best?
\long\def\TeXButton#1#2{#2}%
\long\def\QSubDoc#1#2{#2}%
\def\EXPAND#1[#2]#3{}%
\def\NOEXPAND#1[#2]#3{}%
\def\PROTECTED{}%
\def\LaTeXparent#1{}%
\def\ChildStyles#1{}%
\def\ChildDefaults#1{}%
\def\QTagDef#1#2#3{}%

% Constructs added with Scientific Notebook
\@ifundefined{correctchoice}{\def\correctchoice{\relax}}{}
\@ifundefined{HTML}{\def\HTML#1{\relax}}{}
\@ifundefined{TCIIcon}{\def\TCIIcon#1#2#3#4{\relax}}{}
\if@compatibility
  \typeout{Not defining UNICODE or CustomNote commands for LaTeX 2.09.}
\else
  \providecommand{\UNICODE}[2][]{}
  \providecommand{\CustomNote}[3][]{\marginpar{#3}}
\fi

%
% Macros for style editor docs
\@ifundefined{StyleEditBeginDoc}{\def\StyleEditBeginDoc{\relax}}{}
%
% Macros for footnotes
\def\QQfnmark#1{\footnotemark}
\def\QQfntext#1#2{\addtocounter{footnote}{#1}\footnotetext{#2}}
%
% Macros for indexing.
%
\@ifundefined{TCIMAKEINDEX}{}{\makeindex}%
%
% Attempts to avoid problems with other styles
\@ifundefined{abstract}{%
 \def\abstract{%
  \if@twocolumn
   \section*{Abstract (Not appropriate in this style!)}%
   \else \small 
   \begin{center}{\bf Abstract\vspace{-.5em}\vspace{\z@}}\end{center}%
   \quotation 
   \fi
  }%
 }{%
 }%
\@ifundefined{endabstract}{\def\endabstract
  {\if@twocolumn\else\endquotation\fi}}{}%
\@ifundefined{maketitle}{\def\maketitle#1{}}{}%
\@ifundefined{affiliation}{\def\affiliation#1{}}{}%
\@ifundefined{proof}{\def\proof{\noindent{\bfseries Proof. }}}{}%
\@ifundefined{endproof}{\def\endproof{\mbox{\ \rule{.1in}{.1in}}}}{}%
\@ifundefined{newfield}{\def\newfield#1#2{}}{}%
\@ifundefined{chapter}{\def\chapter#1{\par(Chapter head:)#1\par }%
 \newcount\c@chapter}{}%
\@ifundefined{part}{\def\part#1{\par(Part head:)#1\par }}{}%
\@ifundefined{section}{\def\section#1{\par(Section head:)#1\par }}{}%
\@ifundefined{subsection}{\def\subsection#1%
 {\par(Subsection head:)#1\par }}{}%
\@ifundefined{subsubsection}{\def\subsubsection#1%
 {\par(Subsubsection head:)#1\par }}{}%
\@ifundefined{paragraph}{\def\paragraph#1%
 {\par(Subsubsubsection head:)#1\par }}{}%
\@ifundefined{subparagraph}{\def\subparagraph#1%
 {\par(Subsubsubsubsection head:)#1\par }}{}%
%%%%%%%%%%%%%%%%%%%%%%%%%%%%%%%%%%%%%%%%%%%%%%%%%%%%%%%%%%%%%%%%%%%%%%%%
% These symbols are not recognized by LaTeX
\@ifundefined{therefore}{\def\therefore{}}{}%
\@ifundefined{backepsilon}{\def\backepsilon{}}{}%
\@ifundefined{yen}{\def\yen{\hbox{\rm\rlap=Y}}}{}%
\@ifundefined{registered}{%
   \def\registered{\relax\ifmmode{}\r@gistered
                    \else$\m@th\r@gistered$\fi}%
 \def\r@gistered{^{\ooalign
  {\hfil\raise.07ex\hbox{$\scriptstyle\rm\text{R}$}\hfil\crcr
  \mathhexbox20D}}}}{}%
\@ifundefined{Eth}{\def\Eth{}}{}%
\@ifundefined{eth}{\def\eth{}}{}%
\@ifundefined{Thorn}{\def\Thorn{}}{}%
\@ifundefined{thorn}{\def\thorn{}}{}%
% A macro to allow any symbol that requires math to appear in text
\def\TEXTsymbol#1{\mbox{$#1$}}%
\@ifundefined{degree}{\def\degree{{}^{\circ}}}{}%
%
% macros for T3TeX files
\newdimen\theight
\def\Column{%
 \vadjust{\setbox\z@=\hbox{\scriptsize\quad\quad tcol}%
  \theight=\ht\z@\advance\theight by \dp\z@\advance\theight by \lineskip
  \kern -\theight \vbox to \theight{%
   \rightline{\rlap{\box\z@}}%
   \vss
   }%
  }%
 }%
%
\def\qed{%
 \ifhmode\unskip\nobreak\fi\ifmmode\ifinner\else\hskip5\p@\fi\fi
 \hbox{\hskip5\p@\vrule width4\p@ height6\p@ depth1.5\p@\hskip\p@}%
 }%
%
\def\cents{\hbox{\rm\rlap/c}}%
\def\miss{\hbox{\vrule height2\p@ width 2\p@ depth\z@}}%
%
\def\vvert{\Vert}%           %always translated to \left| or \right|
%
\def\tcol#1{{\baselineskip=6\p@ \vcenter{#1}} \Column}  %
%
\def\dB{\hbox{{}}}%                 %dummy entry in column 
\def\mB#1{\hbox{$#1$}}%             %column entry
\def\nB#1{\hbox{#1}}%               %column entry (not math)
%
\@ifundefined{note}{\def\note{$^{\dag}}}{}%
%

\def\newfmtname{LaTeX2e}
% No longer load latexsym.  This is now handled by SWP, which uses amsfonts if necessary

\ifx\fmtname\newfmtname
  \DeclareOldFontCommand{\rm}{\normalfont\rmfamily}{\mathrm}
  \DeclareOldFontCommand{\sf}{\normalfont\sffamily}{\mathsf}
  \DeclareOldFontCommand{\tt}{\normalfont\ttfamily}{\mathtt}
  \DeclareOldFontCommand{\bf}{\normalfont\bfseries}{\mathbf}
  \DeclareOldFontCommand{\it}{\normalfont\itshape}{\mathit}
  \DeclareOldFontCommand{\sl}{\normalfont\slshape}{\@nomath\sl}
  \DeclareOldFontCommand{\sc}{\normalfont\scshape}{\@nomath\sc}
\fi

%
% Greek bold macros
% Redefine all of the math symbols 
% which might be bolded	 - there are 
% probably others to add to this list

\def\alpha{{\Greekmath 010B}}%
\def\beta{{\Greekmath 010C}}%
\def\gamma{{\Greekmath 010D}}%
\def\delta{{\Greekmath 010E}}%
\def\epsilon{{\Greekmath 010F}}%
\def\zeta{{\Greekmath 0110}}%
\def\eta{{\Greekmath 0111}}%
\def\theta{{\Greekmath 0112}}%
\def\iota{{\Greekmath 0113}}%
\def\kappa{{\Greekmath 0114}}%
\def\lambda{{\Greekmath 0115}}%
\def\mu{{\Greekmath 0116}}%
\def\nu{{\Greekmath 0117}}%
\def\xi{{\Greekmath 0118}}%
\def\pi{{\Greekmath 0119}}%
\def\rho{{\Greekmath 011A}}%
\def\sigma{{\Greekmath 011B}}%
\def\tau{{\Greekmath 011C}}%
\def\upsilon{{\Greekmath 011D}}%
\def\phi{{\Greekmath 011E}}%
\def\chi{{\Greekmath 011F}}%
\def\psi{{\Greekmath 0120}}%
\def\omega{{\Greekmath 0121}}%
\def\varepsilon{{\Greekmath 0122}}%
\def\vartheta{{\Greekmath 0123}}%
\def\varpi{{\Greekmath 0124}}%
\def\varrho{{\Greekmath 0125}}%
\def\varsigma{{\Greekmath 0126}}%
\def\varphi{{\Greekmath 0127}}%

\def\nabla{{\Greekmath 0272}}
\def\FindBoldGroup{%
   {\setbox0=\hbox{$\mathbf{x\global\edef\theboldgroup{\the\mathgroup}}$}}%
}

\def\Greekmath#1#2#3#4{%
    \if@compatibility
        \ifnum\mathgroup=\symbold
           \mathchoice{\mbox{\boldmath$\displaystyle\mathchar"#1#2#3#4$}}%
                      {\mbox{\boldmath$\textstyle\mathchar"#1#2#3#4$}}%
                      {\mbox{\boldmath$\scriptstyle\mathchar"#1#2#3#4$}}%
                      {\mbox{\boldmath$\scriptscriptstyle\mathchar"#1#2#3#4$}}%
        \else
           \mathchar"#1#2#3#4% 
        \fi 
    \else 
        \FindBoldGroup
        \ifnum\mathgroup=\theboldgroup % For 2e
           \mathchoice{\mbox{\boldmath$\displaystyle\mathchar"#1#2#3#4$}}%
                      {\mbox{\boldmath$\textstyle\mathchar"#1#2#3#4$}}%
                      {\mbox{\boldmath$\scriptstyle\mathchar"#1#2#3#4$}}%
                      {\mbox{\boldmath$\scriptscriptstyle\mathchar"#1#2#3#4$}}%
        \else
           \mathchar"#1#2#3#4% 
        \fi     	    
	  \fi}

\newif\ifGreekBold  \GreekBoldfalse
\let\SAVEPBF=\pbf
\def\pbf{\GreekBoldtrue\SAVEPBF}%
%

\@ifundefined{theorem}{\newtheorem{theorem}{Theorem}}{}
\@ifundefined{lemma}{\newtheorem{lemma}[theorem]{Lemma}}{}
\@ifundefined{corollary}{\newtheorem{corollary}[theorem]{Corollary}}{}
\@ifundefined{conjecture}{\newtheorem{conjecture}[theorem]{Conjecture}}{}
\@ifundefined{proposition}{\newtheorem{proposition}[theorem]{Proposition}}{}
\@ifundefined{axiom}{\newtheorem{axiom}{Axiom}}{}
\@ifundefined{remark}{\newtheorem{remark}{Remark}}{}
\@ifundefined{example}{\newtheorem{example}{Example}}{}
\@ifundefined{exercise}{\newtheorem{exercise}{Exercise}}{}
\@ifundefined{definition}{\newtheorem{definition}{Definition}}{}


\@ifundefined{mathletters}{%
  %\def\theequation{\arabic{equation}}
  \newcounter{equationnumber}  
  \def\mathletters{%
     \addtocounter{equation}{1}
     \edef\@currentlabel{\theequation}%
     \setcounter{equationnumber}{\c@equation}
     \setcounter{equation}{0}%
     \edef\theequation{\@currentlabel\noexpand\alph{equation}}%
  }
  \def\endmathletters{%
     \setcounter{equation}{\value{equationnumber}}%
  }
}{}

%Logos
\@ifundefined{BibTeX}{%
    \def\BibTeX{{\rm B\kern-.05em{\sc i\kern-.025em b}\kern-.08em
                 T\kern-.1667em\lower.7ex\hbox{E}\kern-.125emX}}}{}%
\@ifundefined{AmS}%
    {\def\AmS{{\protect\usefont{OMS}{cmsy}{m}{n}%
                A\kern-.1667em\lower.5ex\hbox{M}\kern-.125emS}}}{}%
\@ifundefined{AmSTeX}{\def\AmSTeX{\protect\AmS-\protect\TeX\@}}{}%
%

% This macro is a fix to eqnarray
\def\@@eqncr{\let\@tempa\relax
    \ifcase\@eqcnt \def\@tempa{& & &}\or \def\@tempa{& &}%
      \else \def\@tempa{&}\fi
     \@tempa
     \if@eqnsw
        \iftag@
           \@taggnum
        \else
           \@eqnnum\stepcounter{equation}%
        \fi
     \fi
     \global\tag@false
     \global\@eqnswtrue
     \global\@eqcnt\z@\cr}


\def\TCItag{\@ifnextchar*{\@TCItagstar}{\@TCItag}}
\def\@TCItag#1{%
    \global\tag@true
    \global\def\@taggnum{(#1)}}
\def\@TCItagstar*#1{%
    \global\tag@true
    \global\def\@taggnum{#1}}
%
%%%%%%%%%%%%%%%%%%%%%%%%%%%%%%%%%%%%%%%%%%%%%%%%%%%%%%%%%%%%%%%%%%%%%
%
\def\tfrac#1#2{{\textstyle {#1 \over #2}}}%
\def\dfrac#1#2{{\displaystyle {#1 \over #2}}}%
\def\binom#1#2{{#1 \choose #2}}%
\def\tbinom#1#2{{\textstyle {#1 \choose #2}}}%
\def\dbinom#1#2{{\displaystyle {#1 \choose #2}}}%
\def\QATOP#1#2{{#1 \atop #2}}%
\def\QTATOP#1#2{{\textstyle {#1 \atop #2}}}%
\def\QDATOP#1#2{{\displaystyle {#1 \atop #2}}}%
\def\QABOVE#1#2#3{{#2 \above#1 #3}}%
\def\QTABOVE#1#2#3{{\textstyle {#2 \above#1 #3}}}%
\def\QDABOVE#1#2#3{{\displaystyle {#2 \above#1 #3}}}%
\def\QOVERD#1#2#3#4{{#3 \overwithdelims#1#2 #4}}%
\def\QTOVERD#1#2#3#4{{\textstyle {#3 \overwithdelims#1#2 #4}}}%
\def\QDOVERD#1#2#3#4{{\displaystyle {#3 \overwithdelims#1#2 #4}}}%
\def\QATOPD#1#2#3#4{{#3 \atopwithdelims#1#2 #4}}%
\def\QTATOPD#1#2#3#4{{\textstyle {#3 \atopwithdelims#1#2 #4}}}%
\def\QDATOPD#1#2#3#4{{\displaystyle {#3 \atopwithdelims#1#2 #4}}}%
\def\QABOVED#1#2#3#4#5{{#4 \abovewithdelims#1#2#3 #5}}%
\def\QTABOVED#1#2#3#4#5{{\textstyle 
   {#4 \abovewithdelims#1#2#3 #5}}}%
\def\QDABOVED#1#2#3#4#5{{\displaystyle 
   {#4 \abovewithdelims#1#2#3 #5}}}%
%
% Macros for text size operators:
%
\def\tint{\mathop{\textstyle \int}}%
\def\tiint{\mathop{\textstyle \iint }}%
\def\tiiint{\mathop{\textstyle \iiint }}%
\def\tiiiint{\mathop{\textstyle \iiiint }}%
\def\tidotsint{\mathop{\textstyle \idotsint }}%
\def\toint{\mathop{\textstyle \oint}}%
\def\tsum{\mathop{\textstyle \sum }}%
\def\tprod{\mathop{\textstyle \prod }}%
\def\tbigcap{\mathop{\textstyle \bigcap }}%
\def\tbigwedge{\mathop{\textstyle \bigwedge }}%
\def\tbigoplus{\mathop{\textstyle \bigoplus }}%
\def\tbigodot{\mathop{\textstyle \bigodot }}%
\def\tbigsqcup{\mathop{\textstyle \bigsqcup }}%
\def\tcoprod{\mathop{\textstyle \coprod }}%
\def\tbigcup{\mathop{\textstyle \bigcup }}%
\def\tbigvee{\mathop{\textstyle \bigvee }}%
\def\tbigotimes{\mathop{\textstyle \bigotimes }}%
\def\tbiguplus{\mathop{\textstyle \biguplus }}%
%
%
%Macros for display size operators:
%
\def\dint{\mathop{\displaystyle \int}}%
\def\diint{\mathop{\displaystyle \iint }}%
\def\diiint{\mathop{\displaystyle \iiint }}%
\def\diiiint{\mathop{\displaystyle \iiiint }}%
\def\didotsint{\mathop{\displaystyle \idotsint }}%
\def\doint{\mathop{\displaystyle \oint}}%
\def\dsum{\mathop{\displaystyle \sum }}%
\def\dprod{\mathop{\displaystyle \prod }}%
\def\dbigcap{\mathop{\displaystyle \bigcap }}%
\def\dbigwedge{\mathop{\displaystyle \bigwedge }}%
\def\dbigoplus{\mathop{\displaystyle \bigoplus }}%
\def\dbigodot{\mathop{\displaystyle \bigodot }}%
\def\dbigsqcup{\mathop{\displaystyle \bigsqcup }}%
\def\dcoprod{\mathop{\displaystyle \coprod }}%
\def\dbigcup{\mathop{\displaystyle \bigcup }}%
\def\dbigvee{\mathop{\displaystyle \bigvee }}%
\def\dbigotimes{\mathop{\displaystyle \bigotimes }}%
\def\dbiguplus{\mathop{\displaystyle \biguplus }}%

%%%%%%%%%%%%%%%%%%%%%%%%%%%%%%%%%%%%%%%%%%%%%%%%%%%%%%%%%%%%%%%%%%%%%%%
% NOTE: The rest of this file is read only if amstex has not been
% loaded.  This section is used to define amstex constructs in the
% event they have not been defined.
%
%
\ifx\ds@amstex\relax
   \message{amstex already loaded}\makeatother\endinput% 2.09 compatability
\else
   \@ifpackageloaded{amsmath}%
      {\message{amsmath already loaded}\makeatother\endinput}
      {}
   \@ifpackageloaded{amstex}%
      {\message{amstex already loaded}\makeatother\endinput}
      {}
   \@ifpackageloaded{amsgen}%
      {\message{amsgen already loaded}\makeatother\endinput}
      {}
\fi
%%%%%%%%%%%%%%%%%%%%%%%%%%%%%%%%%%%%%%%%%%%%%%%%%%%%%%%%%%%%%%%%%%%%%%%%
%%
%
%
%  Macros to define some AMS LaTeX constructs when 
%  AMS LaTeX has not been loaded
% 
% These macros are copied from the AMS-TeX package for doing
% multiple integrals.
%
\let\DOTSI\relax
\def\RIfM@{\relax\ifmmode}%
\def\FN@{\futurelet\next}%
\newcount\intno@
\def\iint{\DOTSI\intno@\tw@\FN@\ints@}%
\def\iiint{\DOTSI\intno@\thr@@\FN@\ints@}%
\def\iiiint{\DOTSI\intno@4 \FN@\ints@}%
\def\idotsint{\DOTSI\intno@\z@\FN@\ints@}%
\def\ints@{\findlimits@\ints@@}%
\newif\iflimtoken@
\newif\iflimits@
\def\findlimits@{\limtoken@true\ifx\next\limits\limits@true
 \else\ifx\next\nolimits\limits@false\else
 \limtoken@false\ifx\ilimits@\nolimits\limits@false\else
 \ifinner\limits@false\else\limits@true\fi\fi\fi\fi}%
\def\multint@{\int\ifnum\intno@=\z@\intdots@                          %1
 \else\intkern@\fi                                                    %2
 \ifnum\intno@>\tw@\int\intkern@\fi                                   %3
 \ifnum\intno@>\thr@@\int\intkern@\fi                                 %4
 \int}%                                                               %5
\def\multintlimits@{\intop\ifnum\intno@=\z@\intdots@\else\intkern@\fi
 \ifnum\intno@>\tw@\intop\intkern@\fi
 \ifnum\intno@>\thr@@\intop\intkern@\fi\intop}%
\def\intic@{%
    \mathchoice{\hskip.5em}{\hskip.4em}{\hskip.4em}{\hskip.4em}}%
\def\negintic@{\mathchoice
 {\hskip-.5em}{\hskip-.4em}{\hskip-.4em}{\hskip-.4em}}%
\def\ints@@{\iflimtoken@                                              %1
 \def\ints@@@{\iflimits@\negintic@
   \mathop{\intic@\multintlimits@}\limits                             %2
  \else\multint@\nolimits\fi                                          %3
  \eat@}%                                                             %4
 \else                                                                %5
 \def\ints@@@{\iflimits@\negintic@
  \mathop{\intic@\multintlimits@}\limits\else
  \multint@\nolimits\fi}\fi\ints@@@}%
\def\intkern@{\mathchoice{\!\!\!}{\!\!}{\!\!}{\!\!}}%
\def\plaincdots@{\mathinner{\cdotp\cdotp\cdotp}}%
\def\intdots@{\mathchoice{\plaincdots@}%
 {{\cdotp}\mkern1.5mu{\cdotp}\mkern1.5mu{\cdotp}}%
 {{\cdotp}\mkern1mu{\cdotp}\mkern1mu{\cdotp}}%
 {{\cdotp}\mkern1mu{\cdotp}\mkern1mu{\cdotp}}}%
%
%
%  These macros are for doing the AMS \text{} construct
%
\def\RIfM@{\relax\protect\ifmmode}
\def\text{\RIfM@\expandafter\text@\else\expandafter\mbox\fi}
\let\nfss@text\text
\def\text@#1{\mathchoice
   {\textdef@\displaystyle\f@size{#1}}%
   {\textdef@\textstyle\tf@size{\firstchoice@false #1}}%
   {\textdef@\textstyle\sf@size{\firstchoice@false #1}}%
   {\textdef@\textstyle \ssf@size{\firstchoice@false #1}}%
   \glb@settings}

\def\textdef@#1#2#3{\hbox{{%
                    \everymath{#1}%
                    \let\f@size#2\selectfont
                    #3}}}
\newif\iffirstchoice@
\firstchoice@true
%
%These are the AMS constructs for multiline limits.
%
\def\Let@{\relax\iffalse{\fi\let\\=\cr\iffalse}\fi}%
\def\vspace@{\def\vspace##1{\crcr\noalign{\vskip##1\relax}}}%
\def\multilimits@{\bgroup\vspace@\Let@
 \baselineskip\fontdimen10 \scriptfont\tw@
 \advance\baselineskip\fontdimen12 \scriptfont\tw@
 \lineskip\thr@@\fontdimen8 \scriptfont\thr@@
 \lineskiplimit\lineskip
 \vbox\bgroup\ialign\bgroup\hfil$\m@th\scriptstyle{##}$\hfil\crcr}%
\def\Sb{_\multilimits@}%
\def\endSb{\crcr\egroup\egroup\egroup}%
\def\Sp{^\multilimits@}%
\let\endSp\endSb
%
%
%These are AMS constructs for horizontal arrows
%
\newdimen\ex@
\ex@.2326ex
\def\rightarrowfill@#1{$#1\m@th\mathord-\mkern-6mu\cleaders
 \hbox{$#1\mkern-2mu\mathord-\mkern-2mu$}\hfill
 \mkern-6mu\mathord\rightarrow$}%
\def\leftarrowfill@#1{$#1\m@th\mathord\leftarrow\mkern-6mu\cleaders
 \hbox{$#1\mkern-2mu\mathord-\mkern-2mu$}\hfill\mkern-6mu\mathord-$}%
\def\leftrightarrowfill@#1{$#1\m@th\mathord\leftarrow
\mkern-6mu\cleaders
 \hbox{$#1\mkern-2mu\mathord-\mkern-2mu$}\hfill
 \mkern-6mu\mathord\rightarrow$}%
\def\overrightarrow{\mathpalette\overrightarrow@}%
\def\overrightarrow@#1#2{\vbox{\ialign{##\crcr\rightarrowfill@#1\crcr
 \noalign{\kern-\ex@\nointerlineskip}$\m@th\hfil#1#2\hfil$\crcr}}}%
\let\overarrow\overrightarrow
\def\overleftarrow{\mathpalette\overleftarrow@}%
\def\overleftarrow@#1#2{\vbox{\ialign{##\crcr\leftarrowfill@#1\crcr
 \noalign{\kern-\ex@\nointerlineskip}$\m@th\hfil#1#2\hfil$\crcr}}}%
\def\overleftrightarrow{\mathpalette\overleftrightarrow@}%
\def\overleftrightarrow@#1#2{\vbox{\ialign{##\crcr
   \leftrightarrowfill@#1\crcr
 \noalign{\kern-\ex@\nointerlineskip}$\m@th\hfil#1#2\hfil$\crcr}}}%
\def\underrightarrow{\mathpalette\underrightarrow@}%
\def\underrightarrow@#1#2{\vtop{\ialign{##\crcr$\m@th\hfil#1#2\hfil
  $\crcr\noalign{\nointerlineskip}\rightarrowfill@#1\crcr}}}%
\let\underarrow\underrightarrow
\def\underleftarrow{\mathpalette\underleftarrow@}%
\def\underleftarrow@#1#2{\vtop{\ialign{##\crcr$\m@th\hfil#1#2\hfil
  $\crcr\noalign{\nointerlineskip}\leftarrowfill@#1\crcr}}}%
\def\underleftrightarrow{\mathpalette\underleftrightarrow@}%
\def\underleftrightarrow@#1#2{\vtop{\ialign{##\crcr$\m@th
  \hfil#1#2\hfil$\crcr
 \noalign{\nointerlineskip}\leftrightarrowfill@#1\crcr}}}%
%%%%%%%%%%%%%%%%%%%%%

\def\qopnamewl@#1{\mathop{\operator@font#1}\nlimits@}
\let\nlimits@\displaylimits
\def\setboxz@h{\setbox\z@\hbox}


\def\varlim@#1#2{\mathop{\vtop{\ialign{##\crcr
 \hfil$#1\m@th\operator@font lim$\hfil\crcr
 \noalign{\nointerlineskip}#2#1\crcr
 \noalign{\nointerlineskip\kern-\ex@}\crcr}}}}

 \def\rightarrowfill@#1{\m@th\setboxz@h{$#1-$}\ht\z@\z@
  $#1\copy\z@\mkern-6mu\cleaders
  \hbox{$#1\mkern-2mu\box\z@\mkern-2mu$}\hfill
  \mkern-6mu\mathord\rightarrow$}
\def\leftarrowfill@#1{\m@th\setboxz@h{$#1-$}\ht\z@\z@
  $#1\mathord\leftarrow\mkern-6mu\cleaders
  \hbox{$#1\mkern-2mu\copy\z@\mkern-2mu$}\hfill
  \mkern-6mu\box\z@$}


\def\projlim{\qopnamewl@{proj\,lim}}
\def\injlim{\qopnamewl@{inj\,lim}}
\def\varinjlim{\mathpalette\varlim@\rightarrowfill@}
\def\varprojlim{\mathpalette\varlim@\leftarrowfill@}
\def\varliminf{\mathpalette\varliminf@{}}
\def\varliminf@#1{\mathop{\underline{\vrule\@depth.2\ex@\@width\z@
   \hbox{$#1\m@th\operator@font lim$}}}}
\def\varlimsup{\mathpalette\varlimsup@{}}
\def\varlimsup@#1{\mathop{\overline
  {\hbox{$#1\m@th\operator@font lim$}}}}

%
%Companion to stackrel
\def\stackunder#1#2{\mathrel{\mathop{#2}\limits_{#1}}}%
%
%
% These are AMS environments that will be defined to
% be verbatims if amstex has not actually been 
% loaded
%
%
\begingroup \catcode `|=0 \catcode `[= 1
\catcode`]=2 \catcode `\{=12 \catcode `\}=12
\catcode`\\=12 
|gdef|@alignverbatim#1\end{align}[#1|end[align]]
|gdef|@salignverbatim#1\end{align*}[#1|end[align*]]

|gdef|@alignatverbatim#1\end{alignat}[#1|end[alignat]]
|gdef|@salignatverbatim#1\end{alignat*}[#1|end[alignat*]]

|gdef|@xalignatverbatim#1\end{xalignat}[#1|end[xalignat]]
|gdef|@sxalignatverbatim#1\end{xalignat*}[#1|end[xalignat*]]

|gdef|@gatherverbatim#1\end{gather}[#1|end[gather]]
|gdef|@sgatherverbatim#1\end{gather*}[#1|end[gather*]]

|gdef|@gatherverbatim#1\end{gather}[#1|end[gather]]
|gdef|@sgatherverbatim#1\end{gather*}[#1|end[gather*]]


|gdef|@multilineverbatim#1\end{multiline}[#1|end[multiline]]
|gdef|@smultilineverbatim#1\end{multiline*}[#1|end[multiline*]]

|gdef|@arraxverbatim#1\end{arrax}[#1|end[arrax]]
|gdef|@sarraxverbatim#1\end{arrax*}[#1|end[arrax*]]

|gdef|@tabulaxverbatim#1\end{tabulax}[#1|end[tabulax]]
|gdef|@stabulaxverbatim#1\end{tabulax*}[#1|end[tabulax*]]


|endgroup
  

  
\def\align{\@verbatim \frenchspacing\@vobeyspaces \@alignverbatim
You are using the "align" environment in a style in which it is not defined.}
\let\endalign=\endtrivlist
 
\@namedef{align*}{\@verbatim\@salignverbatim
You are using the "align*" environment in a style in which it is not defined.}
\expandafter\let\csname endalign*\endcsname =\endtrivlist




\def\alignat{\@verbatim \frenchspacing\@vobeyspaces \@alignatverbatim
You are using the "alignat" environment in a style in which it is not defined.}
\let\endalignat=\endtrivlist
 
\@namedef{alignat*}{\@verbatim\@salignatverbatim
You are using the "alignat*" environment in a style in which it is not defined.}
\expandafter\let\csname endalignat*\endcsname =\endtrivlist




\def\xalignat{\@verbatim \frenchspacing\@vobeyspaces \@xalignatverbatim
You are using the "xalignat" environment in a style in which it is not defined.}
\let\endxalignat=\endtrivlist
 
\@namedef{xalignat*}{\@verbatim\@sxalignatverbatim
You are using the "xalignat*" environment in a style in which it is not defined.}
\expandafter\let\csname endxalignat*\endcsname =\endtrivlist




\def\gather{\@verbatim \frenchspacing\@vobeyspaces \@gatherverbatim
You are using the "gather" environment in a style in which it is not defined.}
\let\endgather=\endtrivlist
 
\@namedef{gather*}{\@verbatim\@sgatherverbatim
You are using the "gather*" environment in a style in which it is not defined.}
\expandafter\let\csname endgather*\endcsname =\endtrivlist


\def\multiline{\@verbatim \frenchspacing\@vobeyspaces \@multilineverbatim
You are using the "multiline" environment in a style in which it is not defined.}
\let\endmultiline=\endtrivlist
 
\@namedef{multiline*}{\@verbatim\@smultilineverbatim
You are using the "multiline*" environment in a style in which it is not defined.}
\expandafter\let\csname endmultiline*\endcsname =\endtrivlist


\def\arrax{\@verbatim \frenchspacing\@vobeyspaces \@arraxverbatim
You are using a type of "array" construct that is only allowed in AmS-LaTeX.}
\let\endarrax=\endtrivlist

\def\tabulax{\@verbatim \frenchspacing\@vobeyspaces \@tabulaxverbatim
You are using a type of "tabular" construct that is only allowed in AmS-LaTeX.}
\let\endtabulax=\endtrivlist

 
\@namedef{arrax*}{\@verbatim\@sarraxverbatim
You are using a type of "array*" construct that is only allowed in AmS-LaTeX.}
\expandafter\let\csname endarrax*\endcsname =\endtrivlist

\@namedef{tabulax*}{\@verbatim\@stabulaxverbatim
You are using a type of "tabular*" construct that is only allowed in AmS-LaTeX.}
\expandafter\let\csname endtabulax*\endcsname =\endtrivlist

% macro to simulate ams tag construct


% This macro is a fix to the equation environment
 \def\endequation{%
     \ifmmode\ifinner % FLEQN hack
      \iftag@
        \addtocounter{equation}{-1} % undo the increment made in the begin part
        $\hfil
           \displaywidth\linewidth\@taggnum\egroup \endtrivlist
        \global\tag@false
        \global\@ignoretrue   
      \else
        $\hfil
           \displaywidth\linewidth\@eqnnum\egroup \endtrivlist
        \global\tag@false
        \global\@ignoretrue 
      \fi
     \else   
      \iftag@
        \addtocounter{equation}{-1} % undo the increment made in the begin part
        \eqno \hbox{\@taggnum}
        \global\tag@false%
        $$\global\@ignoretrue
      \else
        \eqno \hbox{\@eqnnum}% $$ BRACE MATCHING HACK
        $$\global\@ignoretrue
      \fi
     \fi\fi
 } 

 \newif\iftag@ \tag@false
 
 \def\TCItag{\@ifnextchar*{\@TCItagstar}{\@TCItag}}
 \def\@TCItag#1{%
     \global\tag@true
     \global\def\@taggnum{(#1)}}
 \def\@TCItagstar*#1{%
     \global\tag@true
     \global\def\@taggnum{#1}}

  \@ifundefined{tag}{
     \def\tag{\@ifnextchar*{\@tagstar}{\@tag}}
     \def\@tag#1{%
         \global\tag@true
         \global\def\@taggnum{(#1)}}
     \def\@tagstar*#1{%
         \global\tag@true
         \global\def\@taggnum{#1}}
  }{}
% Do not add anything to the end of this file.  
% The last section of the file is loaded only if 
% amstex has not been.



\makeatother
\endinput
\usepackage{url}\usepackage[numbers]{natbib}
%-----------------------------------------------------------------------------
% Page Setup
\textheight24cm \textwidth17cm \columnsep6mm
\oddsidemargin-5mm                 % depending on print drivers!
\evensidemargin-5mm                % required margin size: 2cm
\headheight0cm \headsep0cm \topmargin0cm \parindent0cm
                  % delete footer and header
%----------------------------------------------------------------------------
% Environment definitions
%-----------------------------------------------------------------------------
% Using Pictures and tables:
% - Instead "table" write "tablehere" without parameters
% - Instead "figure" write "figurehere " without parameters
% - Please insert a blank line before and after \begin{figurehere} ... \end{figurehere}
%
% CAUTION:   The first reference to a figure/table in the text should be formatted fat.
%




%%%%%%%%%%%%%%%%%%%%%%%%%%%%%%%%%%%%%%%%%%%%%%%%%%%%%%%%%%%%%%%%%%%%%%%%%%%%%%

\makeatother

\begin{document}

\title{Survey on Android Memory Management System}

\maketitle
%

Garza Matteo

Matr. 755295, (matteo.garza@mail.polimi.it)

Tania Suarez Legra

Matr 748927 (tania.suarez@mail.polimi.it)

\hspace{10ex} 

\begin{flushright}
\emph{Report for the master course of Real Time Operative System (RTOS)}\\
 \emph{Reviser: PhD. Patrick Bellasi (bellasi@elet.polimi.it)} 
\par\end{flushright}

Received: April, 01 2011\\
 \hspace{10ex}
\begin{abstract}
Android Operative System\cite{OVERVIEW} is the most diffuse OS in
mobile devices. In this paper we will analyze how Android manages
memory. We discuss in particular about application memory and some
of the most used MMUs used by Android OS.
\end{abstract}
\vspace{4ex}
 \begin{multicols}{2}

%%%%%%%%%%%%%%%%%%%%%%%%%%%%%%%%%%%%%%%%%%%%%%%%%%%%%%%%%%%%%%%%%%%%%%%%%%%%%



\section{Kernel Memory Management}


\subsection{Introduction}

Android\cite{WIKI,OVERVIEW,AndPortal,StatusFeb2012} is a Linux-based
operative system, written in C and C++. Android application software
runs on a framework which includes Java-compatible libraries. Android
uses the Dalvik virtual machine with just-in-time compilation to run
Dalvik dex-code (Dalvik Executable), which is usually translated from
Java bytecode. Figure \ref{Android distribution diffusion (July 2012)}
shows the actual (July 2012)\cite{WIKI} distribution of Android version
between devices with this kernel:

\includegraphics[scale=0.3]{\string"/home/coach/Dropbox/RTOS/roba su github/Android_chart\string".png}

\label{Android distribution diffusion (July 2012)}

We can notice that now the common kernel distributions still use Linux
2.6.x kernel and in particular Android 2.3.

Figure \ref{Android structure (in red, linux kernel parts, in green C++ libraries, in blue Dalvik-interpreted Java applications)}\cite{WIKI}
shows Android OS structure.

\includegraphics[scale=0.3]{\string"/home/coach/Dropbox/RTOS/roba su github/System-architecture\string".jpg}

\label{Android structure (in red, linux kernel parts, in green C++ libraries, in blue Dalvik-interpreted Java applications)}

Android \cite{OVERVIEW} provides some modification to main Linux
kernel, such as an improved power management, ASHMEM virtual memory,
some specific-component drivers, and a low memory killer. The latter's
mission is to free memory when the system run Out of Memory (OOM).

%-----------------------------------------------------------------------------



\subsection{Dalvik Virtual Machine}

Dalvik is the VM in which Android applications are run. Is structured
to work with devices with limited resources:
\begin{itemize}
\item Relatively slow CPU 
\item Small amount of RAM 
\item No swap space 
\end{itemize}
This VM executes Dalvik bytecode, which is compiled from programs
written in the Java. But note that Dalvik VM is not a Java VM (JVM).

Every Android application runs in its own process, with its own instance
of the Dalvik virtual machine, in this way the applications work in
an isolated manner and do not compete with each other. 

Dalvik was written so that a device can run multiple VMs efficiently.
The Dalvik VM executes code in the Dalvik Executable (.dex) format
which is optimized for minimal memory footprint. The VM is register-based,
and runs classes compiled by a Java language compiler that have been
transformed into the .dex format by the included \textquotedbl{}dx\textquotedbl{}
tool. 

An uncompressed .dex file is typically a few percent smaller in size
than a compressed .jar (Java Archive) derived from the same .class
files. 

\includegraphics[bb = 0 0 200 100, draft, type=eps]{/home/coach/android-MMR2012/roba su github/tania repo/asanta96-Ricerca-Android-MMS-2691352/C:/Users/Tania/Desktop/dalvik.png}


\subsection{Low level management and integration with HW resources}

In this part, we discuss about how the memory has managed in Android
devices, focusing on generation of large contiguous buffers. For most
of the releases in Android, it was used PMEM and ASHMEM. These kind
of drivers are way too simple, and was patched with some SoC patches,
such as NVMAP for nVidia Tegra devices and CMEM for TI OMAP ones.
The most important patch was CMA (Contiguous Memory Access), expecially
with DMABUF patch, developed both by Samsung. The most important reason
\cite{StatusFeb2012} is that \textbf{PMEM is not fitted to be used
massively with graphics}. Graphical devices (such as camera) needs
large amount of memory in a very short time (or even in real time),
so the device need to \textbf{avoid memory fragmentation}, that is
space-consuming and mainly time consuming. With the release of Android
4.0 (Ice Cream Sandwich) a brand new driver has released, ION. Thus
is needed to \textbf{unify etherogeneal MMU approaches in a brand
new standardization}. We discuss about differences between ION and
CMA approach, and, in the state-of-art, we discuss of a future integration
between them.

%-----------------------------------------------------------------------------



\subsection{PMEM and ASHMEM}

PMEM (Process MEMory)\cite{AKF} is the \textbf{first memory driver}
implemented on Android devices (since G1). It is used to manage shared
memory regions sufficiently large (from 1 to 16MB).

This regions must be physically contiguous between user space and
kernel drivers (such as GPU, or DSP). It was written specifically
to be used in a very limited hardware platform, and it could be disabled
on x86 architectures. It works in a very simple way: \textbf{it allocs
a bunch of memory at boot time}.\cite{Dec2011Merging}This is dedicated
memory usable for contiguous buffer. As is written above, Pmem is
not suitable for massive use of graphics. The main problem of PMEM
is that \textbf{it exports a device to user space}, giving the applications
the right to alloc direcly buffers to be passed to drivers. Kernel
provides only a low level interface to be used by applications, thus
causing problems of usability and security. \textbf{The majority of
application is written using PMEM approach.}

ASHMEM\cite{ASHMEM} (Android SHared MEMory) is a shared memory allocator
subsystem, similar to POSIX (the classical Linux OS approach), but
with a different behavior. It also gives to the developer an easier
and file-based API. It used \textbf{named memory}, releasable by the
kernel. Apparently, ASHMEM supports low memory devices better than
PMEM, because it could free shared memory units when it is needed. 

\begin{tabular}{|>{\centering}p{0.33\columnwidth}||>{\centering}p{0.4\columnwidth}|}
\hline 
PMEM & ASHMEM\tabularnewline
\hline 
\hline 
Uses physically contiguous addresses & Uses virtual memory\tabularnewline
\hline 
\hline 
The first process who instantiate a memory heap must keep that till
the last one of the users won't free the file descriptor. Thus to
preserve contiguity & Memory is handled by instances (object oriented like). It is managed
by a reference counter\tabularnewline
\hline 
\end{tabular}


\subsection{CMA and DMABUF}

\textbf{CMA (Contiguous Memory Allocator)\cite{CMAdoc} is a well
known framework}, which allows setting up \textbf{machine-specific
configuration} for physically-contiguous memory management. Memory
for devices is then allocated according to that configuration. Differently
from similar framework, it let regions of \textbf{system-reserved
memory to be reused in a transparent way}, letting memory not to be
wasted. When an alloc is instantiated, this framework migrates all
the system page. Thus to build a big chunk of physically contiguous
memory.

Why do an OS have to use chunks of memory?\cite{CMA,RCMA} Because
\textbf{virtual memory tends to fragment pages}. An intensive use
of memory let the system not to be able to find contiguous memory
in a very short time after boot. Recently, the requirement of huge
pages in applications raises, especially for transparent huge pages.
Another question is devices (such as cameras) that needs DMA over
areas physically contiguous. \textbf{CMA reserve an huge area of memory
at boot time}, only for huge request of memory. For every region,
block of pages can be flaggable as three type. 
\begin{itemize}
\item movable : typically, cache pages or anonymous pages, accessed by page
table or page cache radix tree 
\item kernel recallable : they can be given back to the kernel by request. 
\item immovable : these are typically pointer referred pages (such as pages
invoked by a kmalloc()) 
\end{itemize}
The memory manager subsystem \textbf{try to keep movable pages as
near as possible}. Grouping these pages, kernel try to ensure more
and more contiguous free space available for further request. CMA
extends this mechanism. It adds a new type of migration (CMA). Pages
flagged as cma behave like the movable ones, with some differences: 
\begin{itemize}
\item they are ``sticky'', CMA movable pages tends to stay together
\item Their migration type can't be modified by the kernel 
\item In CMA Area, the kernel cannot instantiate pages not movable.
\end{itemize}
In other words, memory flagged as CMA keep available for the rest
of the system with the only restriction to be movable. 

When a driver ask for a huge contiguous allocation of memory, \textbf{CMA
allocator can try to free in his own area some contiguous pages to
create a buffer large as needed}. When the buffer is no longer requested,
memory can be used for other needs. CMA can just take only the needed
amount of memory without worrying about strictly request of alignment.

CMA patches provides a\textbf{ set of function that can prepare regions
of memory and the creation of contest area of a well known size} using
function cm\_alloc and cm\_free to keep and release buffers. \textbf{CMA
must not be invoked by the driver, but from DMA support functions}.
When a driver call a function like dma\_alloc\_coherent(), CMA should
be invoked automatically to satisfying the request. This should work
in normal condition.

One of the issue about CMA is \textbf{how to initially alloc this
area of memory}. Current scheme needs that some of special calls should
be done by the board file system, with a very arm-like approach. The
idea is to do that without board files. The ending result is that
it should be at least one iteration of that patch set before it will
be executed by the mainline. 

%E a un certo punto, scrivo di un altro argomento, DMABUF... un minimo di introduzione servirebbe :P

CMA could be extended letting processes to share buffers, and optimizing
devices using DMA. DMABUF is the DMA buffer sharing framework.

DMA buffers has different request despise of classical allocation
of huge pages.

\begin{tabular}{|>{\centering}p{0.33\columnwidth}|>{\centering}m{0.33\columnwidth}|}
\hline 
DMABUF & Transparent Huge Pages\tabularnewline
\hline 
\hline 
Normally larger than Transparent Huge Pages. 10 Mb.

It could be needed specific memory area, if underlying hardware is
sufficiently ``strange'' & Almost 2Mb large\tabularnewline
\hline 
DMA requires less alignment than THP & 2MB of THP needs 2Mb of Alignment\tabularnewline
\hline 
\end{tabular}

%aggiungere varia roba sui DMABUF


\subsection{ION}

In december 2011,\textbf{ PMEM is marked as deprecated, and then replaced
by ION memory allocator}\cite{ION}. ION is a memory manager that
Google has developed from the 4.0 release of And\textbf{roid (Ice
Cream Sandwich), mainly to resolve the interface issue between different
memory management on each Android device}. In fact, some SoC developer
implemented different memory manager. We can cite some of them:
\begin{itemize}
\item NVMAP, implemented on nVidia Tegra 
\item CMEM\cite{CMEM}, implemented on TI OMAP 
\item HWMEM\cite{HWMEM}, implemented on ST-Ericsonn devices 
\end{itemize}
All this vendor will pass to ION soon.

Besides ION being a memory pool manager, it also enables his clients
to \textbf{share buffers} (so, it works like DMABUF, the DMA buffer
sharing framework). Like PMEM, \textbf{ION manages one or more pools
of memory, some of them instantiated at boot time or from hardware
blocks with specific memory needs}. Some devices like that are GPU,
display controllers and cameras. ION let his pools to be available
as \textbf{heap ION}. Every kind of android device can have different
ION heaps, depending on device memory. \textbf{Physical address and
heap dimension can be returned to the programmer only if the buffer
is physically contiguous}. Buffer can be prepared or deallocated to
be used with DMA, or with virtual kernel addressing. \textbf{Using
a file descriptor, it can be also mapped in the user-space}. There
are three kind of allocable ION heap. Other ones can be defined by
SoC producers (like ION\_HEAP\_TYPE\_SYSTEM\_IOMMU for hardware blocks
equipped with IOMMU driver). 
\begin{itemize}
\item ION\_HEAP\_TYPE\_SYSTEM
\item ION\_HEAP\_TYPE\_SYSTEM\_CONTIG
\item ION\_HEAP\_TYPE\_CARVEOUT : in this case, carveout memory is physically
contiguous and set as boot. 
\end{itemize}
Typically, in the user-space case, libraries uses ION to alloc large
continuous buffers. For instance, camera library can alloc a capture
buffer to be used from the camera device. Once the buffer is fulfilled
with video data, the library gives the buffer to kernel to be processed
by jpeg encoder block. \textbf{A c/c++ program must have access to
'/dev/ion' before it can alloc memory thanks to ION}. He can alloc
data using file descriptors (fd). It can be maximum one client for
user process.

Clients interacts from user-space with ION using \textbf{ioctl() system
interface}. Android processes can share memory using their fd. To
obtain shared buffer, the second user process must obtain a client
handle through a system call open('/dev/ion', O\_RDONLY). \textbf{ION
manage user space client through process PID} (in particular, the
'group leader ' one). Fd will be instantiated pointing at the same
client structure in the kernel. To free a buffer, the second client
must invalidate the mmap() effect, with an explicit call at munmap(),
and the first client must close the fd, calling ION\_IOC\_FREE. This
function decrements the reference counter of the handle. When it reaches
zero, the ion\_handle is destroyed, and the data structure that manages
ION is updated. While managing client calls, ION validates input from
fd, from client and from handler arguments. This validation mechanism
reduce the probability of undesired access and memory leaks. Ion\_buffers
is somewhere similar to DMABUF. Both uses anonymous fd, reference
counted, as shareable objects.

\end{multicols}

\begin{tabular}[t]{|>{\centering}p{0.12\columnwidth}|>{\raggedright}p{0.35\columnwidth}|>{\raggedright}p{0.4\columnwidth}|}
\hline 
 & ION buffers & DMABUF\tabularnewline
\hline 
\hline 
Application level MMU & Alloc and free memory from memory pools in a shareable and trackable
way & It focus on import, export and syncronization in a consisten way with
buffer sharing solution for non arm architectures\tabularnewline
\hline 
Role of Memory manager  & ION replace PMEM as memory pools manager. ION heap lists can be extended
by the device.  & DMABUF is a buffer sharing framework , designed to be integrated with
memory allocator in contiguous DMA mapping framewors, such as CMA.
DMABUF exporters can implement custom allocator.\tabularnewline
\hline 
User Space access control  & ION offers /dev/ion interface to user space program, letting them
to alloc and share buffers. Every user process with ION access can
suspend the system overlapping ION heap. Android chech user and groupID
blocking non authorized access to ION heap & DMABUF offers only kernel API.

Access control is a function of the permissions on device that uses
DMABUF feature\tabularnewline
\hline 
Global Client and Buffer Database.  & ION has a driver associated to /dev/ion. The device structure has
a database that keeps ION buffers allocated, handlers and fd, grouped
by user client and kernel client. ION validates all the client calls
to be valid for database rules. For instance, an handle can't have
two buffers associated. & The debub structure of DMA implements a global hashtable, dma\_entry\_hash,
tracking DMA buffers, but only when kernel is build with CONFIG\_DMA\_API\_

DEBUG option.\tabularnewline
\hline 
Cross- architecture usage & ION usage now is limited on architectures that runs kernel Android & DMABUF usage is cross architecture. DMA mapping redesign let his implementation
in 9 architectures beside the ARM one. \tabularnewline
\hline 
Buffer Syncronization & Ion consider the syncronization problem as an orthogonal problem & DMABUF gives a pair of API for synchronization. Buffer user invokes
dma\_buf\_map\_

attachment() everywhere he desires to use buffer for DMA. Once he
finished using that, signals \textquotedbl{}endOfDMA\textquotedbl{}
to exporter using dma\_buf\_unmap\_

attachment()\tabularnewline
\hline 
Buffer delayed allocation & ION allocs physical memory before the buffer is shared & DMABUF can delay allocation till the first call of dma\_buf\_map\_

attachment(). DMA buffer exporter has the opportunity of scans every
client attachment, collecting all the constraints and choose the most
efficient storage\tabularnewline
\hline 
Integration with Video4

Linux2 API  & Processes that uses these API tends to use PMEM. So, the migration
from PMEM to ION has a relatively small impact. & DMABUF integration with Video4Linux is hard and asked for lots of
modifies in DMABUF. But in a long time that will be a smart choice,
because DMABUF sharing mechanism is fitted for DMA, so it is well
written for CMA and IOMMU. Both of them reduces carveout memory needs
to build an Android smartphone.\tabularnewline
\hline 
\end{tabular}

\begin{multicols}{2}


\section{OOM Killer}


\subsection{Introduction}

Mobile devices become more and more rich of memory over time, due
to Moore's Law. However, there's always a limit over wich memory isn't
available, and a well form kernel needs some politics to free bunch
of memory when needed. Android provides an OOM killer, who kills processes
with some heuristics, letting memory to be used from someone else.
OOM killer mechanism are implemented in most of Linux kernel.

%% da ricontrollare! � copincollato!

Major distribution kernels set the default value of /proc/sys/vm/overcommit\_memory
to zero, which means that \textbf{processes can request more memory
than is currently free in the system}. This is done based on the heuristics
that allocated memory is not used immediately, and that processes,
over their lifetime, also do not use all of the memory they allocate.
Without overcommit, a system will not fully utilize its memory, thus
wasting some of it. Overcommiting memory allows the system to use
the memory in a more efficient way, but at the risk of OOM situations.
Programs who need lots of memory can consume all the system's memory,
stopping the whole system. In such a situation, the OOM-killer kicks
in and identifies the process to be terminated.


\subsection{OOM Killer parameters}

The process to be killed in an out-of-memory situation is selected
\textbf{based on its badness score}. The badness score is reflected
in /proc/<pid>/oom\_score. This value is determined on the basis of
four characteristics:
\begin{itemize}
\item the system loses the minimum amount of work done,
\item recovers a large amount of memory,
\item doesn't kill any innocent process, 
\item and kills the minimum number of processes (if possible limited to
one). 
\end{itemize}
The badness score is computed using 
\begin{itemize}
\item the original memory size of the process, 
\item its CPU time (utime + stime), 
\item the run time (uptime - start time) 
\item and its oom\_adj value. 
\end{itemize}
\textbf{The more memory the process uses, the higher the score. The
longer a process is alive in the system, the smaller the score.}

Any process unlucky enough to be in the swapoff() system call (which
removes a swap file from the system) will be selected to be killed
first. For the rest, the initial memory size becomes the original
badness score of the process. Half of each child's memory size is
added to the parent's score if they do not share the same memory.
Thus forking servers are the prime candidates to be killed. Having
only one \textquotedbl{}hungry\textquotedbl{} child will make the
parent less preferable than the child. Finally, the following heuristics
are applied to save important processes:
\begin{itemize}
\item if the task has nice value above zero, its score doubles 
\item superuser or direct hardware access tasks (CAP\_SYS\_ADMIN, CAP\_SYS\_RESOURCE
or CAP\_SYS\_RAWIO) have their score divided by 4. This is cumulative,
i.e., a super-user task with hardware access would have its score
divided by 16.
\item if OOM condition happened in one cpuset and checked task does not
belong to that set, its score is divided by 8.
\item the resulting score is multiplied by two to the power of oom\_adj
(i.e. points <\textcompwordmark{}<= oom\_adj when it is positive and
points >\textcompwordmark{}>= -(oom\_adj) otherwise). 
\end{itemize}
\textbf{The task with the highest badness score is then selected and
its children are killed}. The process itself will be killed in an
OOM situation when it does not have children.


\subsection{lowmemory driver in Android}

%%da ricontrollare in toto!

Android developers required a greater degree of control over the low
memory situation because the OOM killer does not interfere till late
in the low memory situation, i.e. till all the cache is emptied. Android
need a solution which would start early while the free memory is being
completely depleted. \textbf{So developers introduced the \textquotedbl{}lowmemory\textquotedbl{}
driver}\cite{OOMTAME}, which has multiple thresholds of low memory. 

In a low-memory situation, \textbf{when the first thresholds are met,
background processes are notified of the problem}. They do not exit,
but, instead, save their state. This affects the latency when switching
applications, because the application has to reload on activation.
On further pressure, the \textbf{lowmemory killer kills the non-critical
background processes whose state had been saved} in the previous threshold
and, finally, the foreground applications.

Keeping \textbf{multiple low memory triggers} gives the processes
enough time to free memory from their caches because in an OOM situation,
user-space processes may not be able to run at all. All it takes is
a single allocation from the kernel's internal structures, or a page
fault to make the system run out of memory. An earlier notification
of a low-memory situation could avoid the OOM situation with a little
help from the user space applications which respond to low memory
notifications.

\textbf{Killing processes based on kernel heuristics is not an optimal
solution}, and these new initiatives of offering better control to
the user in selecting the process to be the chosen one to terminate
are steps to a robust design to give more control to the user. 

This approach can be improved in many parts, for instance \cite{OOMarticle}
implementing a more efficient way to select the process to be killed,
such as ordering processes in a red-black tree, improving OOM Killer
response time.

%da segare, in quanto non necessaria?


\subsection{User space OOM control}

%% da leggere e rivedere!
/proc/<pid>/oom\_score is a dynamic value, not so much controllable
and checkable i by the administrator. It is difficult to determine
which process will be killed in case of an OOM condition. \textbf{The
system must let the administrator to modify the score} for every process
created, and for every process which exits. In an attempt to make
OOM-killer policy implementation easier, a \textbf{name-based solution}
was proposed. %by Evgeniy Polyakov. Citare nella bibliografia, se possibile.
With his patch, the process to die first is the one running the program
whose name is found in /proc/sys/vm/oom\_victim. A name based solution
has its limitations:
\begin{itemize}
\item task name is not a reliable indicator of true name and is truncated
in the process name fields. Moreover, symlinks to executing binaries,
but with different names will not work with this approach 
\item This approach can specify only one name at a time, ruling out the
possibility of a hierarchy 
\item There could be multiple processes of the same name but from different
binaries. 
\item The behavior boils down to the default current implementation if there
is no process by the name defined by /proc/sys/vm/oom\_victim. This
increases the number of scans required to find the victim process. 
\end{itemize}
%altra citazione necessaria : "Alan Cox disliked this solution, suggesting that"Another
possible solution is using containers. The patch introduces an OOM
control group (cgroup) with an oom.priority field. \textbf{The process
to be killed is selected from the processes having the highest oom.priority
value.}

This approach could have some trouble, in presence of multiple cpuset.
Consider two cpusets, A and B. If a process in cpuset A has a high
oom.priority value, it will be killed if cpuset B runs out of memory,
even though there is enough memory in cpuset A. 

An interesting outcome of the discussion has been handling OOM situations
in user space. \textbf{The kernel sends notification to user space,
and applications respond by dropping their user-space caches}. In
case the user-space processes are not able to free enough memory,
or the processes ignore the kernel's requests to free memory, the
kernel will kill them.Other hybrid solutions are:
\begin{itemize}
\item the cgroup OOM notifier allows you to attach a task to wait on an
OOM condition for a collection of tasks. This allows userspace to
respond to the condition by dropping caches, adding nodes to a cpuset,
elevating memory controller limits, sending a signal, etc. It can
also defer to the kernel OOM killer as a last resort.
\item /dev/mem\_notify allows you to poll() on a device file and be informed
of low memory events. This can include the cgroup oom notifier behavior
when a collection of tasks is completely out of memory, but can also
warn when such a condition may be imminent. 
\end{itemize}
\bibliographystyle{savetrees}
\bibliography{Report}


\end{multicols} \end{document}
\end{document}
