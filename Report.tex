%% LyX 2.0.2 created this file.  For more info, see http://www.lyx.org/.
%% Do not edit unless you really know what you are doing.
\documentclass[10pt,a4paper]{article}
\usepackage[T1]{fontenc}
\usepackage[latin1]{inputenc}
\pagestyle{empty}
\usepackage{amsmath}
\usepackage{amssymb}

\makeatletter

%%%%%%%%%%%%%%%%%%%%%%%%%%%%%% LyX specific LaTeX commands.
\pdfpageheight\paperheight
\pdfpagewidth\paperwidth

%% Because html converters don't know tabularnewline
\providecommand{\tabularnewline}{\\}

%%%%%%%%%%%%%%%%%%%%%%%%%%%%%% User specified LaTeX commands.
% VDE Template for EUSAR Papers
% Provided by Barbara Lang und Siegmar Lampe
% University of Bremen, January 2002
% English version by Jens Fischer
% German Aerospace Center (DLR), December 2005
% Additional modifications by Matthias Wei{\ss}
% FGAN, January 2009

%-----------------------------------------------------------------------------
% Type of publication

%-----------------------------------------------------------------------------
% Other packets: Most packets may be downloaded from www.dante.de and
% "tcilatex.tex" can be found at (December 2005):
% http://www.mackichan.com/techtalk/v30/UsingFloat.htm
% Not all packets are necessarily needed:
%\usepackage{ngerman} % in german language if required
\usepackage[nooneline,bf]{caption}% Figure descriptions from left margin
\usepackage{times}\usepackage{multicol}\usepackage{epsfig}\input{tcilatex}
%-----------------------------------------------------------------------------
% Page Setup
\textheight24cm \textwidth17cm \columnsep6mm
\oddsidemargin-5mm                 % depending on print drivers!
\evensidemargin-5mm                % required margin size: 2cm
\headheight0cm \headsep0cm \topmargin0cm \parindent0cm
                  % delete footer and header
%----------------------------------------------------------------------------
% Environment definitions
%-----------------------------------------------------------------------------
% Using Pictures and tables:
% - Instead "table" write "tablehere" without parameters
% - Instead "figure" write "figurehere " without parameters
% - Please insert a blank line before and after \begin{figurehere} ... \end{figurehere}
%
% CAUTION:   The first reference to a figure/table in the text should be formatted fat.
%




%%%%%%%%%%%%%%%%%%%%%%%%%%%%%%%%%%%%%%%%%%%%%%%%%%%%%%%%%%%%%%%%%%%%%%%%%%%%%%

\makeatother

\begin{document}

\title{Survey on Android Memory Management System}

\maketitle
%
 Garza Matteo\\
 Matr. 755295, (matteo.garza@mail.polimi.it)\\
 \hspace{10ex} 

\begin{flushright}
\emph{Report for the master course of Real Time Operative System (RTOS)}\\
 \emph{Reviser: PhD. Patrick Bellasi (bellasi@elet.polimi.it)} 
\par\end{flushright}

Received: April, 01 2011\\
 \hspace{10ex}
\begin{abstract}
Android Operative System is the most diffuse OS in lots of devices
(expecially smartphones and tablets). In this paper we will analyze
how Android manages memory on device. We discuss in particular about
application memory and some of the most used MMUs used by Android
OS.
\end{abstract}
\vspace{4ex}
 % Please do not remove or reduce this space here.
\begin{multicols}{2}

%%%%%%%%%%%%%%%%%%%%%%%%%%%%%%%%%%%%%%%%%%%%%%%%%%%%%%%%%%%%%%%%%%%%%%%%%%%%%



\section{Introduction}

TODO

%-----------------------------------------------------------------------------



\subsection{Memory Management}

In this part, we discuss about how the memory has managed in Android
devices. For most of the releases in Android, it was used PMEM and
ASHMEM. These kind of libraries was too simple, and was patched with
some SoC patches, such as NVMAP for nVidia Tegra devices and CMEM
for TI OMAP ones. The most important patch was CMA (Contiguous Memory
Access), expecially with DMABUF patch. With the release of Android
4.0 (Ice Cream Sandwich) a brand new library has released, ION. We
discuss about differences between ION and CMA approach, and, in the
state-of-art, we discuss of a future integration between them.

%-----------------------------------------------------------------------------



\subsection{PMEM and ASHMEM}

PMEM (Process MEMory) is the first memory driver implemented on Android
devices (since G1). It is used to manage shared memory regions sufficiently
large (from 1 to 16MB).

This regions must be physically contiguous between user space and
kernel drivers (such as GPU, or DSP). It was written specifically
to be used in a very limited hardware platform, and it could be disabled
on x86 architectures.

ASHMEM (Android SHared MEMory) is a shared memory allocator subsystem,
similar to POSIX, but with a different behavior. It also gives to
the developer an easier and file-based API. It used named memory,
releasable by the kernel. Apparently, ASHMEM supports low memory devices
better than PMEM, because it could free shared memory units when it
is needed. 

\begin{figure}
\caption{}
\end{figure}
\begin{table}
\begin{tabular}{|c|c|}
\hline 
PMEM & ASHMEM\tabularnewline
\hline 
\hline 
Uses physically contiguous addresses & Uses virtual memory\tabularnewline
\hline 
The first process who instantiate a memory heap must keep them till
the last of the users won't release file descriptor, thus to keep
contiguity & Memory is handled by instances (object oriented like). It is managed
by a reference counter\tabularnewline
\hline 
\end{tabular}

\caption{PMEM vs ASHMEM}
\end{table}
%%%%%%%%%%%%%%%%%%%%%%%%%%%%%%%%%%%%%%%%%%%%%%%%%%%%%      PMEM VS ASHMEM should be here



\subsection{CMA e DMABUF}

CMA (Contiguous Memory Allocator) is a well known patch that let the
device to alloc big chunk of memory after the system has booted. Differently
from similar framework, it let regions of system-reserved memory to
be reused in a transparent way, letting memory not to be wasted. When
an alloc is instantiated, this framework migrates all the system page.
Thus to build a big chunk of physically contiguous memory.

Why do an OS have to use chunks of memory? Because virtual memory
tends to fragment pages. An intensive use of memory let the system
not able to find contiguous memory in a very short time after boot.
Recently, the requirement of huge pages in applications raises, especially
for transparent huge pages. Another question is devices (such as cameras)
that needs DMA over areas physically contiguous. CMA reserve an huge
area of memory at boot time, only for huge request of memory. For
every region, block of pages can be flaggable as three type. 
\begin{itemize}
\item movable : typically, cache pages or anonymous pages, accessed by page
table or page cache radix tree 
\item kernel recallable : they can be given back to the kernel by request. 
\item immovable : these are typically pointer referred pages (such as pages
invoked by a kmalloc()) 
\end{itemize}
The memory manager subsystem try to keep movable pages as near as
possible. Grouping these pages, kernel try to ensure more and more
contiguous free space available for further request. CMA extends this
mechanism. It adds a new type of migration (CMA). Pages flagged as
cma behave like the movable ones, with some differences: 
\begin{itemize}
\item they are ``sticky'' 
\item Their migration type can't be modified by the kernel 
\item In CMA Area, the kernel cannot instantiate pages not movable.
\end{itemize}
In other words, memory flagged as CMA keep available for the rest
of the system with the only restriction to be movable. 

When a driver ask for a huge contiguous allocation of memory, CMA
allocator can try to free in his own area some contiguous pages to
create a buffer large as needed. When the buffer is no longer requested,
memory can be used for other needs. CMA can just take only the needed
amount of memory without worrying about strictly request of alignment.

DMA buffers has different request despise of huge pages.

%%%%%%%%%%%%%%%%%%%%%%%%%%%%%%%%%%%%%%%%%%%%%%%%%%%%%    qui la tabella di confronto


CMA patches provides a set of function that can prepare regions of
memory and the creation of contest area of a well known size using
function cm\_alloc and cm\_free to keep and release buffers. CMA must
not be invoked by the driver, but from DMA support functions. When
a driver call a function like dma\_alloc\_coherent(), CMA should be
invoked automatically to satisfying the request. This should work
in normal condition.

One of the issue about CMA is how to initially alloc this area of
memory. Current scheme needs that some of special calls should be
done by the board file system, with a very arm-like approach. The
idea is to do that without board files. The ending result is that
it should be at least one iteration of that patch set before it will
be executed by the mainline. 


\subsection{ION}

In december 2011, PMEM is marked as deprecated, and then replaced
by ION memory allocator. ION is a memory manager that Google has developed
from the 4.0 release of Android (Ice Cream Sandwich), mainly to resolve
the interface issue between different memory management between different
Android device. In fact, some SoC developer implemented different
memory manager. We can cite some of them:
\begin{itemize}
\item NVMAP, implemented on nVidia Tegra 
\item CMEM, implemented on TI OMAP 
\item HWMEM, implemented on ST-Ericsonn devices 
\end{itemize}
All this vendor will pass to ION soon

Besides ION being a memory pool manager, it also enables his clients
to share buffers (so, it works like DMABUF, the DMA buffer sharing
framework). Like PMEM, ION manages one or more pools of memory, some
of them instantiated at boot time or from hardware blocks with specific
memory needs. Some devices like that are GPU, display controllers
and cameras. ION let his pools to be available as heap ION. Every
kind of android device can have different ION heaps, depending on
device memory. Phisical address and heap dimension can be returned
to the programmer only if the buffer is physically contiguous. Buffer
can be prepared or deallocated to be used with DMA, or with virtual
kernel addressing. Using a file descriptor, it can be also mapped
in the user-space. There are three kind of allocable ION heap. Other
ones can be defined by SoC producers (like ION\textbackslash{}\_HEAP\textbackslash{}\_TYPE\textbackslash{}\_SYSTEM\textbackslash{}\_IOMMU
for hardware blocks equipped with IOMMU driver). 
\begin{itemize}
\item ION\textbackslash{}\_HEAP\textbackslash{}\_TYPE\textbackslash{}\_SYSTEM
\item ION\textbackslash{}\_HEAP\textbackslash{}\_TYPE\textbackslash{}\_SYSTEM\textbackslash{}\_CONTIG
\item ION\textbackslash{}\_HEAP\textbackslash{}\_TYPE\textbackslash{}\_CARVEOUT
: in this case, carveout memory is physically contiguous and set as
boot. 
\end{itemize}
Typically, in the user-space case, libraries uses ION to alloc large
continuous buffers. For instance, camera library can alloc a capture
buffer to be used from the camera device. Once the buffer is fulfilled
with video data, the library gives the buffer to kernel to be processed
by jpeg encoder block. A c/c++ program must have access to '/dev/ion'
before it can alloc memory thanks to ION. He can alloc data using
file descriptors (fd). It can be maximum one client for user process.

Clients interacts from user-space with ION using ioctl() system interface.
Android processes can share memory using their fd. To obtain shared
buffer, the second user process must obtain a client handle through
a system call open('/dev/ion', O\textbackslash{}\_RDONLY). ION manage
user space client through process PID (in particular, the 'group leader
' one). Fd will be instantiated pointing at the same client structure
in the kernel. To free a buffer, the second client must invalidate
the mmap() effect, with an explicit call at munmap(), and the first
client must close the fd, calling ION\_IOC\_FREE. This function decrements
the reference counter of the handle. When it reaches zero, the ion\_handle
is destroyed, and the data structure that manages ION is updated. 

While managing client calls, ION validates input from fd, from client
and from handler arguments. This validation mechanism reduce the probability
of undesired access and memory leaks. Ion\_buffers is somewhere similar
to DMABUF. 


\section{The Second Section}

Lorem ipsum dolor sit amet, consectetur adipiscing elit. Aenean magna.
Nunc non ante eget nibh condimentum tempor. Nullam ullamcorper lectus
eget mauris. Nam neque orci; rhoncus at, pulvinar quis, elementum
sit amet, turpis. Mauris posuere nisi ut justo. Morbi non lorem vitae
mauris interdum faucibus. Vestibulum ut sapien in augue faucibus fringilla.
Vestibulum ante ipsum primis in faucibus orci luctus et ultrices posuere
cubilia Curae; Etiam vestibulum fringilla libero. Curabitur libero
diam, hendrerit sit amet, ornare eget, imperdiet vel, purus!

%-----------------------------------------------------------------------------



\subsection{The first subsection of the second \protect \\
 Section}

Lorem ipsum dolor sit amet, consectetur adipiscing elit. Nam consectetur
ante at eros. Vestibulum mi nisi, venenatis sollicitudin, tempus sed,
auctor id, tortor. Fusce orci. Duis tellus arcu, euismod sed, consequat
sit amet, elementum vel, mauris. Curabitur leo diam; dapibus quis,
condimentum vitae, dignissim ut, diam. Nulla et nulla eget elit volutpat
sagittis.

%-----------------------------------------------------------------------------



\subsection{The second subsection of the second \protect \\
 Section}

Lorem ipsum dolor sit amet, consectetur adipiscing elit. Mauris eget
mauris. Nulla facilisi. Ut condimentum tempor eros? Integer metus
mauris, consectetur sit amet, tempor a, facilisis eu, nisl. Vestibulum
at turpis. Ut vitae tortor pretium nisl vestibulum blandit. Nulla
nibh urna, semper et, elementum at, mattis ut, nisi! Cum sociis natoque
penatibus et magnis dis parturient montes, nascetur ridiculus mus.
Morbi vel ligula eget lacus convallis venenatis. Aliquam lacinia tincidunt
felis. Ut dui.

% We suggest the use of JabRef for editing your bibliography file (Report.bib)
 \bibliographystyle{splncs}
\bibliography{Report}


\end{multicols} 
\end{document}
